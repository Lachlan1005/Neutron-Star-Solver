\documentclass{article}
\usepackage{graphicx} % Required for inserting images
\usepackage{amsmath}
\usepackage{amsfonts}
\usepackage{amssymb}
\usepackage{bm}
\usepackage{physics}
\usepackage{fancyhdr}
\usepackage{pgfplots}
\usepackage{siunitx}
\usepackage{braket}
\usepackage{mhchem}
\usepackage{chemfig}
\usepackage{gensymb}
\newcommand{\ve}{\mathbf}
\newcommand{\pa}{\partial}
\newcommand{\la}{\langle}
\newcommand{\ra}{\rangle}

\title{Proofs and Derivations}
\author{Neutron Stars for Undergraduates}
\date{March 2025}

\begin{document}
\maketitle
\section*{Equation 1}
We know that the outer face of the small bit of the star (as shown in figure 1) experience less 
force than the inner face, since the outer face is a distance $r+dr$ away and
 the inner only $r$ away from the center. Therefore we would expect the gravity differential $dF$ to be negative.  
Suppose the mass of the block is $dm$, hence $dF$ is given by
\begin{align}
    dF=-\frac{GMdm}{r^2}
\end{align}
If the star hass mass density $\rho$, we can rewrite (1) according to the details provided in figure 1.
\begin{align}
    dF=-\frac{GM\rho A \ dr}{r^2}
\end{align}
Using the definition of pressure, we obtain the solution. 
\begin{align}
    dp=\frac{dF}{A}=-\frac{GM\rho \ dr}{r^2}\implies \frac{dp}{dr}=-\frac{GM\rho}{r^2}
\end{align}
We can rewrite the solution in terms of energy. Since $E=mc^2$, dividing both sices by $V$ gives 
$\epsilon=\rho c^2$ where $\epsilon$ is the energy density. Hence (3) becomes 
\begin{align}
    \boxed{\frac{dp}{dr}=-\frac{GM\epsilon}{c^2r^2}}
\end{align}
\section*{Equation 2 and 3}
The definition of density gives the following 
\begin{align}
    dM=\rho\ dV
\end{align}
The usual $dV$ for radial symmetry is $dV=4\pi r^2 dr$, hence (5) becomes 
\begin{align}
    dM=4\pi\rho r^2\ dr=\frac{4\pi\epsilon r^2}{c^2}\ dr\implies \boxed{\frac{dM}{dr}=\frac{4\pi\epsilon r^2}{c^2}}
\end{align}
Using the separated form of (6) and integrating both sides, we obtain 
\begin{align}
    M=\int_0^r\frac{4\pi\epsilon r'^2}{c^2}dr'
\end{align}
Where the variable change $r\mapsto r'$ was made since $r$ has to be an integration bound.

\section*{Equation 10}
To obtain the total energy of a set of electrons with the same spin, we integrate over the energy of every single possible state, 
\begin{align}
    E_{total}=\int E\ dN\implies \epsilon=\int E\ dn
\end{align}
Where we have divided both sides by $V$, and used the definitions $\epsilon=E/V$ and $n=N/V\implies dn=dN/V$.
We replace energy by Einstein's formula, where momentum is denoted by $k$ (they use this convention in the paper). 
\begin{align}
    \epsilon=\int\sqrt{k^2c^2+m^2c^4}dn
\end{align}
From equation (6) in the paper, we replace $dn$ with the expression in terms of $dk$, giving 
\begin{align}
    \epsilon=\int \sqrt{k^2c^2+m^2c^4}\frac{4\pi k^2}{(2\pi\hbar)^3}\ dk
\end{align}
Since we are integrating over all possible states, the integration bounds are naturally from 0 to the fermi momentum 
$k_f$. Additionally, since there are 2 spin states per electron, we must multiply the whole expression by 2.
\begin{align}
    \boxed{\epsilon=\frac{8\pi}{(2\pi\hbar)^3}\int_0^{k_f}\sqrt{k^2c^2+m^2c^4}\ k^2\ dk}
\end{align}
\section*{Equation 12}
From the definition that $dU=dQ-pdV$, we set $dQ$ to 0 since there is no change in temperature, 
and thus obtain $dU=-pdV$, hence 
\begin{align}
    p=-\frac{dU}{dV}
\end{align}
Since $U$ is the total energy of the nucleons, we can write $\epsilon=U/V$. Now, replacing the above, we have 
\begin{align}
    p=-\frac{d(\epsilon V)}{dV}=-(\frac{d\epsilon}{dV}V+\epsilon)
\end{align}
Let us examine the first term, 
\begin{align}
    \frac{d\epsilon}{dV}=\frac{d\epsilon}{dn}\frac{dn}{dV}=\frac{d\epsilon}{dn}\frac{d}{dV}(N/V)=-\frac{d\epsilon}{dn}\frac{N}{V^2}
\end{align}
Where we have used the definition $n=N/V$. Thus our equation (13) becomes 
\begin{align}
    p=-(-\frac{d\epsilon}{dn}\frac{N}{V^2}V+\epsilon)=\boxed{\frac{d\epsilon}{dn}n-\epsilon}
\end{align}
\section*{Equation 13}
Let us return to our expression for $\epsilon$
\begin{align}
     {\epsilon=\frac{8\pi}{(2\pi\hbar)^3}\int_0^{k_f}\sqrt{k^2c^2+m^2c^4}\ k^2\ dk}
\end{align}
For convienience, we shall define $\epsilon'$ as follows 
\begin{align}
    \epsilon'&=\int_0^{k_f}\sqrt{k^2c^2+m^2c^4}\ k^2\ dk\\
\end{align}
Now, we integrate by parts. Differentiating the energy part and integrating the momentum part, we obtain 
\begin{align}
    &=\sqrt{k^2c^2+m^2c^4}\int_0^{k_f}k^2\ dk -\int_0^{k_f}\frac{d}{dk}\sqrt{k^2c^2+m^2c^4}\int k^2\ dk\ dk\\ 
    &=\frac{1}{3}k_f^3\sqrt{k^2c^2+m^2c^4}-\int_0^{k_f}\frac{1}{3}k^3\frac{c^2k}{\sqrt{k^2c^2+m^2c^4}}dk\\
    &=\frac{1}{3}k_f^3\sqrt{k^2c^2+m^2c^4}-\frac{1}{3}\int_0^{k_f}\frac{c^2k^4}{\sqrt{k^2c^2+m^2c^4}}dk
\end{align}
Thus, multiplying by the original coefficients present in our (16), we obtain the expression for $\epsilon$. 
\begin{align}
    \epsilon=\frac{8\pi k_f^3}{(2\pi\hbar)^3}\sqrt{k^2c^2+m^2c^4}-\frac{8\pi}{3(2\pi\hbar)^3}\int_0^{k_f}\frac{c^2k^4}{\sqrt{k^2c^2+m^2c^4}}dk
\end{align}
If we multiply our (9) by 2, we get the total energy density for all the electrons. Differentiating that expression WRT $n$ gives 
the following 
\begin{align}
    \frac{d\epsilon}{dn}=2\sqrt{k^2c^2+m^2c^4}
\end{align}
If we multiply this by the paper's definition of $n$ in its (7), we obtain 
\begin{align}
    n\frac{d\epsilon}{dn}=\frac{8\pi k_f^3}{(2\pi\hbar)^3}\sqrt{k^2c^2+m^2c^4}
\end{align}
Hence, we can substitute everything into our (15) and obtain 
\begin{align}
    p&=\frac{8\pi k_f^3}{(2\pi\hbar)^3}\sqrt{k^2c^2+m^2c^4}-(\frac{8\pi k_f^3}{(2\pi\hbar)^3}\sqrt{k^2c^2+m^2c^4}-\frac{1}{3}\int_0^{k_f}\frac{c^2k^4}{\sqrt{k^2c^2+m^2c^4}}dk)\\
\implies& \boxed{p=\frac{8\pi c^2}{3(2\pi\hbar)^3}\int_0^{k_f}\frac{c^2k^4}{\sqrt{k^2+m^2c^4}}dk}
\end{align}
\section*{Equation 14}
The following is the second line of the paper's (13)
\begin{align}
    p=\frac{m^4c^5}{3\pi^2\hbar^3}\int_0^{k_f/mc}\frac{u^4}{\sqrt{u^2+1}}\ du
\end{align}
In the relativistic case, the particles have extremely high momenta, thus $k_f>>m$, thus the ratio $u=k_f/mc>>1$, meaning $u^2+1\approx u^2$. 
Thus the above becomes 
\begin{align}
    p&=\frac{m^4c^5}{3\pi^2\hbar^3}\int_0^{k_f/mc}\frac{u^4}{\sqrt{u^2}}\ du=\frac{m^4c^5}{3\pi^2\hbar^3}\int_0^{k_f/mc}u^3\ du\\
    &=\frac{m^4c^5}{3\pi^2\hbar^3}\frac{k_f^4}{4(mc)^4}
\end{align}
Recall the definition of $k_f$ from the paper. Substituting that into our (28) gives 
\begin{align}
    p=\frac{\hbar c}{12\pi^2}(\frac{3\pi^2\rho}{m_N}\frac{Z}{A})^{4/3}
\end{align}
Recall that $\epsilon\approx \rho c^2$ where electron mass contribution is minimal. Thus,  we can rewrite the pressure as 
\begin{align}
    p\approx\frac{\hbar c}{12\pi^2}(\frac{3\pi^2}{m_Nc^2}\frac{Z}{A})^{4/3}\epsilon^{4/3}
\end{align}
Now, we define the following parameter 
\begin{align}
    K_{rel}\equiv\frac{\hbar c}{12\pi^2}(\frac{3\pi^2}{m_Nc^2}\frac{Z}{A})^{4/3}
\end{align}
And thus our (30) beocmes the final expression 
\begin{align}
    \boxed{p\approx K_{rel}\epsilon^{4/3}}
\end{align}
\section*{Equation 15}
Now we consider the case where the particles are slow moving such that relativistic effects can 
be neglected. In this case, $k_f<<m$, hence $u=k_f/mc\approx0$. Hence our (26) becomes 
\begin{align}
    p&=\frac{m^4c^5}{3\pi^2\hbar^3}\int_0^{k_f/mc}\frac{u^4}{\sqrt{1}}\ du=\frac{m^4c^5}{3\pi^2\hbar^3}\frac{k_f^5}{5(mc)^5}
\end{align}
Again, recalling the definition of Fermi momentum and using $\epsilon=\rho c^2$, the above can be rewritten as 
\begin{align}
    p=\frac{\hbar^2}{15\pi^2m}(\frac{3\pi^2 Z}{m_N c^2 A})^{5/3}\epsilon^{5/3}
\end{align}
We again define yet another parameter 
\begin{align}
    K_{nonrel}\equiv\frac{\hbar^2}{15\pi^2m}(\frac{3\pi^2 Z}{m_N c^2 A})^{5/3}
\end{align}
Our (34) is thus in its final desired form 
\begin{align}
    \boxed{p=K_{nonrel}\epsilon^{5/3}}
\end{align}
\section*{Equation 18}
From the start we have the following
\begin{align}
    \frac{dp}{dr}=-\frac{GM\epsilon}{c^2r^2}=-\frac{G\epsilon M_{\odot}M/M_{\odot}}{c^2r^2}=-\frac{GM_{\odot}}{c^2}\frac{\epsilon M/M_{odot}}{r^2}
\end{align}
Using the definitions of $R_0$ and $\bar{M}$ from the paper, we obtain the following 
\begin{align}
    \boxed{\frac{dp}{dr}=-R_0\frac{\epsilon\bar{M}}{r^2}}
\end{align}
\section*{Equation 23}
According to the definitions in the paper before equation 23, the normalised pressure $\bar{p}$ can be written in 
differential form in terms of $dp/dr$ as follows 
\begin{align}
    \frac{d\bar{p}}{dr}=\frac{d(p/\epsilon_0)}{dr}=\frac{1}{\epsilon_0}\frac{dp}{dr}=-R_0\frac{\epsilon/\epsilon_0\bar{M}}{r^2}=-R_0\frac{\bar{\epsilon}\bar{M}}{r^2}
\end{align}
Replacing the $\bar{\epsilon}$ with the paper's (22), the pressure equation is now in terms of pressure. This simplifies the problem a lot.
\begin{align}
    \frac{d\bar{p}}{dr}=-R_0\frac{(\bar{p}/\bar{K})^{1/\gamma}\bar{M}}{r^2}=-\frac{R_0}{\bar{K}^{1/\gamma}}\frac{\bar{p}^{1/\gamma}\bar{M}}{r^2}
\end{align}
We now define the parameter 
\begin{align}
    \alpha\equiv\frac{R_0}{\bar{K}^{1/\gamma}}
\end{align}
Thus, our (41) becomes the following  
\begin{align}
    \boxed{\frac{d\bar{p}}{dr}=-\alpha\frac{\bar{p}^{1/\gamma}\bar{M}}{r^2}}
\end{align}
\section*{Equation 26}
From our (26), the mass equation reads
\begin{align}
    \frac{dM}{dr}=\frac{4\pi\epsilon r^2}{c^2}
\end{align}
Dividing both sides by $M_{\odot}$ and recognising that $\epsilon=\epsilon_0\bar{\epsilon}$ gives the following 
\begin{align}
    \frac{d\bar{M}}{dr}=\frac{4\pi\bar{\epsilon}\epsilon_0}{c^2M_{\odot}}r^2
\end{align}
Now, using the paper's (22) to replace $\bar{\epsilon}$, we obtain the above in terms of pressure  instead of energy density
\begin{align}
    \frac{d\bar{M}}{dr}=\frac{4\pi\epsilon_0}{c^2M_{\odot}\bar{K}^{1/\gamma}}\bar{p}^{1/\gamma}r^2
\end{align}
We define another parameter as follows 
\begin{align}
    \beta\equiv\frac{4\pi\epsilon_0}{c^2M_{\odot}\bar{K}^{1/\gamma}}
\end{align}
Now, the mass equation is much simpler, reading 
\begin{align}
    \boxed{\frac{d\bar{M}}{dr}=\beta\bar{p}^{1/\gamma}r^2}
\end{align}
\section*{Equation 52}
For a collection of particles, we previously reasoned that we can find the total energy density by 
integrating their energies across the number density 
\begin{align}
    \epsilon=\int_0^N E\ dn
\end{align}
Since the chemical potential $\mu$ is given by $d\epsilon/dn$, we can find it as 
\begin{align}
    \mu=\frac{d\epsilon}{dn}=\frac{d}{dn}\int_0^N E\ dn=E\bigg|_0^{k_f}
\end{align}
Using Eistein's equation, it is obvious that the $N$th particle will have momentum equal to 
the Fermi momentum, hence 
\begin{align}
    \mu=\sqrt{c^2k_f^2+m^2c^2}
\end{align}
We shall now perform a unit transformation, transitioning into natural units where $c=1$. Hence, 
(51) can now be rewritten into a simpler form 
\begin{align}
    \boxed{\mu=\sqrt{k_f^2+m^2}}
\end{align}
\section*{Equation 69*}
Here we shall derive how the kinetic
 energy term in the paper's (69) can be rewritten in the form $\la E_f^0\ra u^{2/3}$. 
 The kinetic energy term  reads 
 \begin{align}
    \la E_f\ra&=\frac{3}{5}\frac{\hbar^2 k_f^2}{2m}
 \end{align}
 Here $m$ refers to the nucleon mass. We can expand $k_f$ according to its definition. 
 \begin{align}
    k_f=\hbar(\frac{3\pi^2\rho}{m}\frac{Z}{A})^{1/3}=\hbar(\frac{3\pi^2}{m}\frac{nmA}{Z}\frac{Z}{A})^{1/3}=\hbar(3\pi^2n)^{1/3}
 \end{align}
 Since $u=n/n_0$, we can multiply our (54) my $n_0/n_0$ to get it in terms of $u$ 
 \begin{align}
    k_f=\hbar(3\pi^2n_0 n/n_0)^{1/3}=\hbar(3\pi^2n_0)^{1/3}u^{1/3}
 \end{align}
 Notice that $\hbar(3\pi^2n_0)^{1/3}$ is simply the Fermi energy when $n=n_0$, in other words, the 
 number density is at the equillibrium number density. Hence, if we denote this fermi energy by $k_f^0$, we obtain 
 \begin{align}
    k_f=k_f^0u^{1/3}\implies k_f^2=(k_f^0)^2u^{2/3}
 \end{align}
 Now, recall the definition of average Fermi energy, 
 \begin{align}
    \la E_f\ra=\frac{3}{5}\frac{\hbar^2}{2m}k_f^2=\frac{3}{5}\frac{\hbar^2}{2m}(k_f^0)^2u^{2/3}
 \end{align}
 Notice again that everything before $u^{2/3}$ is simply a restatement of the Fermi energy when $n=n_0$, 
 which we shall denote as $\la E_f^0\ra$. We arrive finally at the desired form 
 \begin{align}
    \boxed{\la E_f\ra=\la E_f^0\ra u^{2/3}}
 \end{align}
 And hence the energy per nucleon can be written as 
 \begin{align}
    \frac{E}{A}=\frac{\epsilon}{n}=m+\la E_f^0\ra u^{2/3}+\frac{A}{2}u+\frac{B}{\sigma+1}u^{\sigma}
 \end{align}
 \section*{Equation 77}
 We can find the pressure using our previously derived result 
 \begin{align}
    p=n^2\frac{d(\epsilon/n)}{dn}=\frac{n^2}{n_0^2}n_0^2 \frac{d(\epsilon/n)}{du}\frac{du}{dn}=u^2n_0^2\frac{d(\epsilon/n)}{du}\frac{du}{dn}
 \end{align}
 Let us evaluate the derivatives. The first one reads
 \begin{align}
    \frac{d(\epsilon/n)}{du}&=\frac{d}{du}(m+\la E_f^0\ra u^{2/3}+\frac{A}{2}u+\frac{B}{\sigma+1}u^{\sigma})\\ 
    &=\frac{2}{3}\la E_f^0\ra u^{-1/3}+\frac{A}{2}+\frac{B\sigma}{\sigma+1}u^{\sigma-1}
 \end{align}
 Now on to the second one 
 \begin{align}
    \frac{du}{dn}=\frac{d}{dn}(\frac{n}{n_0})=\frac{1}{n_0}
 \end{align}
 Hence our (60) can now be expressed as 
 \begin{align}
    p&=u^2n_0^2(\frac{2}{3}\la E_f^0\ra u^{-1/3}+\frac{A}{2}+\frac{B\sigma}{\sigma+1}u^{\sigma-1})\frac{1}{n_0}\\ 
 \end{align}
 And finally distributing the factor of $u^2$ at the start, we arrive at the final expression 
 \begin{align}
    \boxed{p=n_0(\frac{2}{3}\la E_f^0\ra u^{5/3}+\frac{A}{2}u^2+\frac{B\sigma}{\sigma+1}u^{\sigma+1})}
 \end{align}
 \section*{Equation 81}
 The number of either the neutrons or the protons can be expressed as follows 
 \begin{align}
    n_{np}=\frac{1\pm\alpha}{2}n
 \end{align}
 Where plus is for neutron and minus is for proton. 
 Hence the Fermi momentum for either is simply 
 \begin{align}
    k_{f}=\hbar(3\pi^2 n_{np})^{1/3}
 \end{align}
 If we square and multiply it by the neutron/proton number $n_{np}$, we arrive at 
 \begin{align}
    k_f^2n_{np}=\hbar(3\pi^2)^{2/3}n_{np}^{5/3}=\hbar(3\pi^2)^{2/3}(\frac{1\pm\alpha}{2})^{5/3}n^{5/3}
 \end{align}
Hence, the energy density for either a neutron or a proton can be expressed by 
\begin{align}
    \epsilon_{np}=\frac{3}{5}\frac{k_f^2}{2m}n_{np}=\frac{3}{5}\frac{\hbar^2}{2m}(3\pi^2)^{2/3}(\frac{1\pm\alpha}{2})^{5/3}n^{5/3}
\end{align}
We now add together the energy density of the neutrons (chooseing $\pm$ to be $+$) and the energy density of protons (choosing $\pm$ to be $-$).
\begin{align}
    \epsilon=\frac{3}{5}\frac{\hbar^2}{2m}\frac{(3\pi^2)^{2/3}}{2^{5/3}}n^{5/3}[(1+\alpha)^{5/3}+(1-\alpha)^{5/3}] 
\end{align}
Now we rewrite the above into a more familliar form 
\begin{align}
    \epsilon=\frac{3}{5}\frac{\hbar^2}{2m}(\frac{3\pi^2 n}{2})^{2/3}\frac{n}{2}[(1+\alpha)^{5/3}+(1-\alpha)^{5/3}] 
\end{align}
Notice that everything before the factor of $n/2$ is the average Fermi energy. If we rewrite it as that, we arrive at 
\begin{align}
    \boxed{\epsilon=\frac{n}{2}\la E_f\ra[(1+\alpha)^{5/3}+(1-\alpha)^{5/3}] }
\end{align}
\section*{Equation 85}
Using our (73), we can find the energy density for symmetric nuclear matter, where $\alpha=0$. This gives 
\begin{align}
    \epsilon=n\la E_f\ra
\end{align} 
We can thus find the difference in energy density between symmetric and asymmetric nuclear matter as follows
\begin{align}
    \Delta\epsilon&=\frac{n}{2}\la E_f\ra[(1+\alpha)^{5/3}+(1-\alpha)^{5/3}]-n\la E_f\ra\\ 
    &=n\la{E_f}\ra[\frac{1}{2}[(1+\alpha)^{5/3}+(1-\alpha)^{5/3}]-1]
\end{align}
We shall consider the folloing expansions 
\begin{align}
    (1+\alpha)^{5/3}&=\sum_{i=0}^{\infty} {5/3 \choose i}\alpha^i\\ 
    (1-\alpha)^{5/3}&=\sum_{i=0}^{\infty} {5/3 \choose i}(-1)^i\alpha^i
\end{align}
This is an exact representations for all values of $\alpha$, since $-1\leq\alpha\leq1$ is the 
range of valid $\alpha$ and just so happens to also be 
the interval of convergence of these power series. Hence the sum in our (76) becomes 
\begin{align}
    (1+\alpha)^{5/3}+(1-\alpha)^{5/3}&= \sum_{i=0}^{\infty} {5/3 \choose i}\alpha^i+\sum_{i=0}^{\infty} {5/3 \choose i}(-1)^i\alpha^i\\ 
    &=\sum_{i=0}^{\infty} {5/3 \choose i}\alpha^i [1+(-1)^i]\\ 
    &=2+0+\frac{10}{9}\alpha^2+0+\frac{10}{243}\alpha^4+O(\alpha^6)\\
    &\approx2+\frac{10}{9}\alpha^2+\frac{10}{243}\alpha^4
\end{align}
Where in the last step the higher order terms $O(\alpha^6)$ are deemed to be
 negligible. Notice that the expansion is even in $\alpha$. This is a result from the isospin symmetry, 
 requiring that nuclear energy must be symmetric under interchange of protons and neutrons.   Our (76) hence becomes 
\begin{align}
    \Delta\epsilon&=n\la{E_f}\ra[\frac{1}{2}[2+\frac{10}{9}\alpha^2+\frac{10}{243}\alpha^4]-1]\\ 
    &=n\la E_f\ra(\frac{5}{9}\alpha^2+\frac{5}{243}\alpha^4)
\end{align}
Notice that 243 is divisible by 27, and hence we can factor out the whole $5\alpha^2/9$ term out from our (84), 
hence giving the final form 
\begin{align}
    \boxed{\Delta\epsilon=\frac{5}{9}n\la{E_f}\ra\alpha^2(1+\frac{\alpha^2}{27})}
\end{align}
\section*{Normalising the TOV Equations}
The pressure equation for the TOV pair of equations is as follows 
\begin{align}
    \dv{p}{r}=-\frac{G\epsilon M}{c^2r^2}(1+\frac{p}{\epsilon})(1+\frac{4\pi}{c^2}\frac{r^3p}{M})(1-\frac{2GM}{c^2r})^{-1}
\end{align}
We seek to transform this into an equation containing only the normalised pressure $\bar{p}$ and the normalised mass $\bar{M}$ and relevant constants, 
and we will rely heavily on the definitions introduced in page 895. We see that 
\begin{align}
    \dv{\bar{p}}{r}=\dv{(p/\epsilon_0)}{r}=\frac{1}{\epsilon_0}\dv{p}{r}
\end{align}
If we absorb the coefficient into the first term, it becomes 
\begin{align}
    -\frac{G\epsilon/\epsilon_0 M}{c^2r^2}=-\alpha\frac{\bar{M}\bar{p}^{1/\gamma}}{r^2}
\end{align}
Where we skipped over the exact steps since it is the same as deriving the paper's 
(43). Feel free to 
revisit that derivation if this seems unfamilliar. Now, the second factor can also be rewritten as follows
\begin{align}
    1+\frac{p}{\epsilon}=1+\frac{p/\epsilon_0}{\epsilon/\epsilon_0}=1+\frac{\bar{p}}{\bar{\epsilon}}=1+\bar{p}(\frac{\bar{K}}{\bar{p}})^{1/\gamma}=1+\bar{K}^{1/\gamma}\bar{p}^{1-1/\gamma}
\end{align}
Onto the third factor. Keep in mind the definitions of the normalised variables. 
\begin{align}
    1+\frac{4\pi}{c^2}\frac{r^3p}{M}=1+\frac{4\pi}{c^2}\frac{r^3\bar{p}/\epsilon_0}{\bar{M}M_{\odot}}=1+\frac{4\pi}{M_{\odot}c^2\epsilon_0}\frac{\bar{p}}{\bar{M}}r^3=1+\beta\frac{\bar{K}^{1/\gamma}}{\epsilon_0^2}\frac{\bar{p}}{\bar{M}}r^3
\end{align}
Finally, the terms inside the brackets of the last factor reads
\begin{align}
    1-\frac{2GM}{c^2r}=1-2\frac{GM_{\odot}}{c^2}\frac{\bar{M}}{r}=1-2R_0\frac{\bar{M}}{r}
\end{align}
Putting it all together, the pressure equation reads as follows 
\begin{align}
\boxed{    \dv{p}{r}=-\alpha\frac{\bar{M}\bar{p}^{1/\gamma}}{r^2}(1+\bar{K}^{1/\gamma}\bar{p}^{1-1/\gamma})(1+\beta\frac{\bar{K}^{1/\gamma}}{\epsilon_0^2}\frac{\bar{p}}{\bar{M}}r^3)(1-2R_0\frac{\bar{M}}{r})^{-1}
}\end{align}
Since the mass equation only consists of adding up all the masses from the center to the surface of th e
star, nothing changes the equation, hence it is identical to the newtonian approach, 
which is the same 
the same as our (48), reading 
\begin{align}
    \boxed{\dv{\bar{M}}{r}=\beta\bar{p}^{1/\gamma}r^2}
\end{align}
\section*{The Lane-Emden Equations}
Recall that at the very start, we derived the following set of equations for the pressure 
and mass of a star. 
\begin{align}
    \frac{dp}{dr}&=-\frac{GM\rho}{r^2}\\
    \frac{dM}{dr}&=4\pi\rho r^2
\end{align}
We can try to eliminate $M$ from the top equation to get this pair to reduce to a single, 
second order ODE. To do so, we need to get the $M$ in our (94) to be in terms of $dM/dr$, 
which is whats given in our (95). Notice that if we simple differentiated our (94) WRT $r$, we would have to apply the 
produce rule three times, since the quantity $M\rho/r^2$ have factors all dependent on $r$. The 
easiest way to get $M$ by itself is to multiply both sides by $r^2/\rho$. After that we can differentiate 
both sides WRT $r$. 
\begin{align}
    \frac{r^2}{\rho}\frac{dp}{dr}=-GM\implies \frac{d}{dr} 
    (\frac{r^2}{\rho}\frac{dp}{dr} ) =-G\frac{dM}{dr}
\end{align} 
Substituting in the $dM/dr$ from our (95) and dividing both sides by $r^2$, we obtain 
\begin{align}
    \boxed{\frac{1}{r^2}\frac{d}{dr}(\frac{r^2}{\rho}\frac{dp}{dr})=-4\pi G\rho}
\end{align}
The solutions of $p$ and $\rho$ are well known to be given in the following form 
\begin{align}
    p&=K\rho_0^{1+1/n}\phi^{n+1} \\
    \rho&=\rho_0\phi^n
\end{align}
Where $\phi=\phi(r)$ is a function of $r$, and is currently unknown. Therefore, our objective 
is to rewrite our (97) in terms of $\phi$, so that we only have to solve for one function instead of two in our equation.
We begin by examining the part inside of the parenthesis. 
\begin{align}
    \frac{r^2}{\rho}\frac{dp}{dr}=\frac{r^2}{\rho}\frac{dp}{d\phi}\frac{d\phi}{dr}
\end{align}
Since we know the relation between $p$ and $\phi$ from our (98), we can easily find that 
\begin{align}
    \frac{dp}{d\phi}=K\rho_0^{1+1/n}(n+1)\phi^n
\end{align}
Substituting this inside our expression in (100) and also recognising that $\rho$ is given by our equation (99), we obtain 
\begin{align}
    \frac{r^2}{\rho}\frac{dp}{d\phi}\frac{d\phi}{dr}=\frac{r^2}{\rho_0\phi^n}K\rho_0^{1+1/n}(n+1)\phi^n\frac{d\phi}{dr}=r^2K\rho_0^{1/n}(n+1)\frac{d\phi}{dr}
\end{align}
Hence, our (97) becomes 
\begin{align}
    \frac{1}{r^2}\frac{d}{dr}(r^2K\rho_0^{1/n}(n+1)\frac{d\phi}{dr})=-4\pi G\rho_0\phi^n
\end{align}
Where we have substituted the $\rho$ on the right hand side for the expression in our (99).
Now, if we gather all the constants to outside of the brackets on the left hand side, we obtain 
\begin{align}
    \frac{K\rho_0^{1/n-1}(n+1)}{4\pi G}\frac{1}{r^2}\frac{d}{dr}(r^2\frac{d\phi}{dr})=-\phi^n
\end{align}
For simplicity, we now define a constant 
\begin{align}
    \alpha^2\equiv \frac{K\rho_0^{1/n-1}(n+1)}{4\pi G}
\end{align}
The square on $\alpha$ is because $r$ is always squared in our equation, so that we can define 
a new variable 
\begin{align}
    \xi\equiv\frac{r}{\alpha}
\end{align}
Now, we can do a change of variables in our equation, where the following changes in the operators has to be made 
\begin{align}
    \frac{d}{dr}&=\frac{d}{d\xi}\frac{d\xi}{dr}=\frac{1}{\alpha}\frac{d}{d\xi} 
\end{align}
Where we have recognised that from our (106), $d\xi/dr=1/\alpha$. Now, our (104) becomes 
\begin{align}
    \frac{\alpha^2}{r^2}\frac{d}{dr}(r^2\frac{d\phi}{dr})&=-\phi^2
    \implies \frac{1}{\xi^2}\frac{1}{\alpha}\frac{d}{d\xi}(\alpha^2\xi^2\frac{1}{\alpha}\frac{d\phi}{d\xi})=-\phi^n
\end{align}
Notice that the $\alpha$ on the left hand side all cancels out. Moving everything to the left hand side, we are left with the classic form 
of the Lane-Emden equation. 
\begin{align}
    \boxed{\frac{1}{\xi^2}\frac{d}{d\xi}(\xi^2\frac{d\phi}{d\xi})+\phi^n=0}
\end{align}
\end{document}

